\documentclass{amsart}

\usepackage{defs}
\usepackage{draftwatermark}
\SetWatermarkScale{8}
\usepackage{autonum} % Only number referenced eqn's

% For the Cahier du GERAD.
%\renewcommand{\year}{2020}     % If you don't want the current year.
\newcommand{\cahiernumber}{00}  % Insert your Cahier du GERAD number.

\begin{document}

\title[Regularized Constrained Least Squares]{%
  A Regularized Interior-Point Method for Constrained Linear Least Squares
}

%    author one information
\author[M. Dehghani]{Mohsen Dehghani}
\address{%
  GERAD, Montr\'eal, Canada
}
\email{\mailto{mohsen.dehghani@gerad.ca}}
\thanks{}

%    author two information
\author[D. Orban]{Dominique Orban}
\address{%
  GERAD and
  Mathematics and Industrial Engineering Department \\
  \'Ecole Polytechnique, Montr\'eal, Canada
}
\urladdr{\http{www.gerad.ca/~orban}}
\email{\mailto{dominique.orban@gerad.ca}}
\thanks{Research partially supported by NSERC Discovery Grant 299010-04}

\subjclass[2010]{90C06, 90C20, 90C30, 90C51, 90C53, 90C55, 65F10, 65F50}

\keywords{%
  linear least squares, linear constraints, symmetric quasi-definite
  system, interior-point method, primal-dual regularization, proximal point,
  augmented Lagrangian}

\date{\today}

\begin{abstract}
  We propose an infeasible interior-point algorithm for constrained linear
  least-squares problems based on the primal-dual regularization of convex
  programs of \cite{friedlander-orban-2012}. At each iteration, the sparse
  LDL${}^{\mathrm{T}}$ factorization of a symmetric quasi-definite matrix is
  computed. This  matrix is shown to be uniformly bounded and
  nonsingular. We establish conditions under which a solution of the original
  problem is recovered. Regularization allows us to dispense with the
  assumption that the active gradients are linearly independent. Although the
  implementation described here is factorization based, it paves the way to a
  matrix-free implementation in which a regularized unconstrained linear
  least-squares problem is solved at each iteration. We report computational
  experience and illustrate the potential advantages of our approach.
\end{abstract}

\pagestyle{myheadings}

\maketitle

%\tableofcontents
%\listoftodos\relax   % Must appear after toc.

\section{Introduction}

We are concerned with the constrained linear least-squares problem in standard
form:
\begin{equation}
  \label{eq:lsq}
  \minimize{x \in \R^n} \ c^T x + \half \|Ax - d\|^2 \quad
  \st \ Bx = b, \ x \geq 0,
\end{equation}
where $c \in \R^n$, $A \in \R^{p \times n}$, $d \in \R^p$, $B \in \R^{m \times
n}$, $b \in \R^m$ and inequalities are understood componentwise. It is
typically assumed that $p > n$ and $m < n$ but the approach proposed in this
paper allows us to do away with these restrictions. If $A = 0$, \eqref{eq:lsq}
reduces to the linear programming problem in standard form. In all cases,
\eqref{eq:lsq} is a convex quadratic program. Numerous applications give rise
to \eqref{eq:lsq}, including signal decomposition by basis pursuit
\citep{chen-donoho-saunders-1998}, $\ell_1$-regularized linear least squares
\citep{kim-koh-lustig-boyd-gorinevsky-2007}, compressed sensing
\citep{donoho-2006},  and machine learning \citep{koh-kim-boyd-2007}.

An interior-point method applied directly to \eqref{eq:lsq} might suffer
several difficulties. Firstly, the matrix $A^T \! A$, which may be rather
dense and ill conditioned, will appear explicitly in the Newton step
computation. Secondly, numerical instabilities will arise if the constraint
matrix $B$ does not have full row rank. We remove the first difficulty in two
different ways that lead to two slightly different implementations. The second
difficulty disappears if we consider the following regularization of
\eqref{eq:lsq} proposed by \cite{friedlander-orban-2012}:
\begin{equation}
  \begin{aligned}
  \label{eq:lsq-reg}
    \minimize{x \in \R^n, w \in \R^m} \quad &
                              c^T x + \half \|Ax - d\|^2 +
                              \half \rho \|x - x_k\|^2 +
                              \half \delta \|w + y_k\|^2 \\
    \st \quad & B x + \delta w = b, \ x \geq 0,
  \end{aligned}
\end{equation}
where $\rho > 0$ and $\delta > 0$ are regularization parameters, $x_k$ and
$y_k$ are the current approximations of the optimal primal variables and
Lagrange multipliers, respectively, and $w$ are auxiliary variables playing
the role of a constraint residual.

In this paper, we specialize the interior-point framework of
\cite{friedlander-orban-2012} and apply it to \eqref{eq:lsq-reg} with
ultimately constant regularization parameters. At each iteration, a step is
computed by solving a large and sparse symmetric quasi-definite linear system
\citep{vanderbei-1995}. Contrary to most interior-point implementations,
partial block elimination is not applied to this system to reduce it to the
so-called \textit{augmented system} form or to the \textit{normal equations}.
Instead, a similarity transformation is applied that guarantees that the system
remains uniformly bounded and nonsingular throughout the iterations and in the
limit,  provided strict complementarity is satisfied at a solution. We establish
global convergence under weak assumptions. In particular, no assumption on the
rank of $B$ or $A$. A distinctive feature of the regularization
\eqref{eq:lsq-reg} is that it allows to be recovered a solution of \eqref{eq:lsq} in
many situations,  not a solution of a perturbed problem. In addition,
\eqref{eq:lsq-reg} is never solved to optimality for fixed values of $\rho$,
$\delta$, $x_k$ and $y_k$. Instead, it is used to compute a single Newton step
before attention turns to the next regularized subproblem.

In \eqref{eq:lsq-reg}, the \textit{primal} regularization term $\half
\rho \|x - x_k\|^2$ serves the dual purpose of regularizing $A$ whenever it
is rank deficient and simplifying the implementation of the interior-point
method in the presence of free variables. The \textit{dual} regularization term
$\half \delta \|w + y_k\|^2$ regularizes $B$ whenever it is rank--deficient.

The implementation proposed below relies on a sparse $LDL^T$ factorization
of the symmetric quasi-definite  matrix. This factorization may be
obtained at lower cost than the symmetric indefinite factorization, such as
that of \cite{duff-2004}, and typically yields sparser factors. Its stability
on symmetric quasi-definite systems has been analyzed by
\cite{gill-saunders-shinnerl-1996}.

Many applications only provide $A$ and $B$ in the form of linear operators
instead of explicit matrices. Iterative methods specialized to symmetric
quasi-definite systems have been  proposed recently by \cite{arioli-orban-2012}.
Our algorithm paves the way to a matrix-free implementation using such iterative
methods. This yields an elegant framework in which an unconstrained
regularized linear least-squares problem must be solved at each iteration.

Our analysis and implementation differ from those of
\cite{friedlander-orban-2012} in several respects. First, the
linear systems used in the definition of the Newton steps are larger, sparser
and tailored to the special structure of \eqref{eq:lsq}. If strict
complementarity holds at a solution, they also have uniformly bounded condition
number. Second, our approach illustrates how to apply the primal-dual
regularization of \cite{friedlander-orban-2012} selectively, leaving some
variables and some constraints untouched. This has the benefit of exploiting
the structure of the problem at hand.

\subsection{Related Research}

Most of the literature concentrates on the box-constrained variant of
\eqref{eq:lsq}, i.e., with no linear equality constraints.
\cite{portugal-judice-vicente-1994} compare infeasible active-set-type methods
with interior-point algorithms and conclude that it is beneficial to use the
normal equations formulation of the step-finding subproblem in the
interior-point method. \cite{bellavia-macconi-morini-2006} propose an inexact
Newton scheme under the assumption that $A$ has full rank but specifically
target problems failing to satisfy strict complementarity at the solution.
\cite{bellavia-gondzio-morini-2008} regularize the linear systems arising in
the method of \cite{bellavia-macconi-morini-2006} with the aim of decreasing
their condition number. The resulting linear systems are subsequently partially
eliminated, preconditioned and solved with an iterative method.
\cite{bellavia-gondzio-morini-2011} propose and compare an interior-point
method and a method derived from that of \cite{barzilai-borwein-1988}.

Active-set methods for the box-constrained problems have also been actively
researched. Among others, \cite{bierlaire-toint-tuyttens-1991} describe three
methods for reasonably well-conditioned problems, two of which are inspired by
trust-region methods. \cite{friedlander-2007} implements a dual-metric method
in which search directions are linear combinations of the Newton direction and
a scaled gradient direction.

Finally, background and pointers to more literature may be found in the seminal
books of \cite{lawson-hanson-1995} and \cite{bjorck-1996}.


\subsection{Notation}

The notation $X$ and $Z$ is used to denote the diagonal matrices $\diag(x)$ and
$\diag(z)$. The vector $e$ denotes the vector of all ones of appropriate
dimension. The notation $\|\cdot\|$ denotes the Euclidian norm throughout. The
$i$-th component of a vector $x$ is denoted $[x]_i$ while the value of $x$
at the $k$-th iteration of a process is denoted $x_k$. For a given
positive definite matrix $M$, the $M$-norm is defined as $\|x\|_M^2 = x^T M x$.
The notation $\blkdiag(A_1, \ldots, A_k)$ denotes a block-diagonal matrix
having the blocks $A_1$ through $A_k$ consecutively on the diagonal. Whenever
a block $A_j$ is an identity block, its size is dictated by the context. For
two related sequences $\{\alpha_k\}$ and $\{\beta_k\}$ of positive numbers, we
write $\alpha_k = O(\beta_k)$ if there exists a constant $C > 0$ such that
$\alpha_k \leq C \beta_k$ for all sufficiently large $k$. We write $\alpha_k =
\Theta(\beta_k)$ if $\alpha_k = O(\beta_k)$ and $\beta_k = O(\alpha_k)$.

\section{Background and Preliminaries}

As a convex quadratic program, the dual of \eqref{eq:lsq} may be written as the
constrained linear least-squares problem
\begin{equation}
  \label{eq:lsq-dual}
  \begin{aligned}
    \maximize{x,y,z} \quad &
      b^T y - (A^T d)^T x - \half \|Ax - d\|^2 \\
    \st \quad & B^T y + z - A^T \! A x = c - A^T d, \ z \geq 0,
  \end{aligned}
\end{equation}
where $y \in \R^m$ and $z \in \R^n$ are the vectors of Lagrange multipliers
associated with the equality constraints and bounds of \eqref{eq:lsq},
respectively.

\cite{friedlander-orban-2012} justify the regularization \eqref{eq:lsq-reg} of
\eqref{eq:lsq} as an application of the proximal method of multipliers of
\cite{rockafellar-1976}. It consists of the addition of a proximal-point term
$\half \rho \|x - x_k\|^2$,  which we will refer to as a \textit{primal
regularization} term, and of augmented Lagrangian terms consisting of the
objective term $\half \delta \|w + y_k\|^2$ and the constraint
residual term $\delta w$,   which we will collectively refer to as a \textit{dual
regularization} term. The regularized problem \eqref{eq:lsq-reg} is still a
convex quadratic program and its dual may be written as the regularized
constrained linear least-squares problem
\begin{equation}
  \label{eq:lsq-reg-dual}
  \begin{aligned}
    \maximize{x, s, y, z} \quad &
    b^T y - (A^T d)^T x - \half \|Ax - d\|^2 -
    \half \delta \|y - y_k\|^2 - \half \rho \|s + x_k\|^2 \\
    \st \quad & B^T y + z - A^T \! A x - \rho s = c - A^T d, \ z \geq 0,
  \end{aligned}
\end{equation}
where $s = x - x_k$ are auxiliary variables playing the same role as $w$ in
\eqref{eq:lsq-reg}. The strength of this regularization approach is that
\eqref{eq:lsq-reg-dual} coincides with the primal-dual regularization of
\eqref{eq:lsq-dual}.

For convenience, we let $u := (x, s, w, y, z)$ and we define the function
\begin{equation}
  \label{eq:def-F}
  F_k(u; \rho, \delta, \tau) :=
  \begin{bmatrix}
    c - A^T d - B^T y - z + A^T \! A x + \rho s \\
    \rho x - \rho (s + x_k) \\
    \delta y - \delta (w + y_k) \\
    Bx + \delta w - b \\
    X z - \tau e
  \end{bmatrix},
\end{equation}
where $\tau \geq 0$ is a parameter. Using this notation, the common necessary
and sufficient optimality conditions of \eqref{eq:lsq-reg} and
\eqref{eq:lsq-reg-dual} may be written compactly as
\[
  F_k(u; \rho, \delta, 0) = 0,
  \qquad
  (x, z) \geq 0.
\]
Note also that additionally setting  $\rho = \delta = 0$ recovers the optimality
conditions of the original primal-dual pair \eqref{eq:lsq} and
\eqref{eq:lsq-dual}.

An interior-point method applied to the primal-dual pair \eqref{eq:lsq-reg} and
\eqref{eq:lsq-reg-dual} iteratively seeks approximate solutions to the
nonlinear system
\[
  F_k(u; \rho, \delta, \tau_k) = 0, \quad (x,z) > 0,
\]
for a sequence of positive parameters $\{\tau_k\} \downarrow 0$. For each
fixed value of $\tau_k$, a Newton step $\Delta u$ is computed from the current
approximation $u_k$ as the solution of the linear system
\[
  \nabla_u F_k(u_k; \rho, \delta, \tau_k) \Delta u =
  -F_k(u_k; \rho, \delta, \tau_k),
\]
where the Jacobian is given by
\[
  \nabla_u F_k(u_k; \rho, \delta, \tau_k) =
  \begin{bmatrix}
    A^T \! A  & \phantom{-}\rho I   & & -B^T & -I \\
    \rho I & -\rho I & &      & \\
    & & -\delta I & \delta I & & \\
    B & & \phantom{-}\delta I & & & \\
    Z & & & & X
  \end{bmatrix}.
\]
Of particular concern is that the matrix $A^T \! A$ is (nearly) dense if $A$ has a
(nearly) dense row. One way to circumvent this difficulty is to introduce
$\xi := d - A \Delta x - Ax$. An equivalent way to write the previous system is
then
\[
  \begin{bmatrix}
    & A^T & \phantom{-}\rho I & & -B^T & -I \\
    A & I & & & & \\
    & & & \vdots & &
  \end{bmatrix}
  \begin{bmatrix}
    \Delta x \\ \xi \\ \vdots
  \end{bmatrix}
  =
  \begin{bmatrix}
    -c + B^T y + z - \rho s \\
    d - A x \\
    \vdots
  \end{bmatrix},
\]
where the ellipses indicate that the rest of the system is unchanged. This
system is larger but sparser and does away with the matrix-matrix product
$A^T \! A$.

A more natural way to give rise to the previous sparse Jacobian is to
systematically transform every problem of the form \eqref{eq:lsq} to the form
\begin{equation}
  \label{eq:lsq-r}
  \begin{aligned}
    \minimize{x, r} \quad & c^T x + \half \|r\|^2 \\
    \st \quad & Bx = b, \quad Ax + r = d, \quad x \geq 0,
  \end{aligned}
\end{equation}
whose dual may be written
\begin{equation}
  \label{eq:lsq-r-dual}
  \begin{aligned}
    \maximize{x, r, y, z} \quad & b^T y - (A^T d)^T x - \half \|r\|^2 \\
    \st \quad & B^T y + z + A^T r = c, \quad
                Ax + r = d, \quad z \geq 0.
  \end{aligned}
\end{equation}
Note that the quadratic term in $r$ is the same in the objective function of
both the primal and dual problem. An important difference from standard
quadratic programming is the appearance of the linear term in $x$ in the
dual objective. We return to the importance of this term in
\S\ref{sec:implementation}.
It is interesting to note that the constraints $Ax + r = d$, defining the
residual $r$, appear in both the primal and the dual problem. This simple fact
turns out to guide our choice of regularization.

We now consider the following regularization of \eqref{eq:lsq-r}:
\begin{equation}
  \label{eq:lsq-r-reg}
  \begin{aligned}
    \minimize{x, r, w} \quad &
      c^T x + \half \|r\|^2 +
      \half \rho \|x - x_k\|^2 +
      \half \delta \|w + y_k\|^2 \\
    \st \quad &
      Bx + \delta w = b, \quad Ax + r = d, \quad x \geq 0.
  \end{aligned}
\end{equation}
This regularization differs from \eqref{eq:lsq-reg} in two respects. First,
no primal regularization term is added for the variables $r$ because the
objective function of \eqref{eq:lsq-r} is already strictly convex in $r$.
Second, the constraints $Ax + r = d$ are not regularized because they already
have full row rank. Since the variables $r$ do not appear elsewhere in the
constraints, the equality constraints of \eqref{eq:lsq-r-reg} have full row
rank. No harm would be done in regularizing $Ax + r = d$ although it would
result in a larger system. The dual of \eqref{eq:lsq-r-reg} may be stated as
\begin{equation}
  \label{eq:lsq-r-dual-reg}
  \begin{aligned}
    \maximize{x, r, y, s} \quad &
      b^T y - (A^T d)^T x - \half \|r\|^2 -
      \half \delta \|y - y_k\|^2 -
      \half \rho \|s + x_k\|^2 \\
    \st \quad &
      A^T r + B^T y + z - \rho s = c, \quad
      Ax + r = d, \quad
      z \geq 0,
  \end{aligned}
\end{equation}
where we introduced the auxiliary variables $s = x - x_k$. Because the dual
\eqref{eq:lsq-r-dual-reg} also features the full-rank constraints $Ax + r = d$,
it offers an additional elegant justification for omitting the primal
regularization term for the variables $r$ in \eqref{eq:lsq-r-reg}. Indeed,
regularizing those constraints in \eqref{eq:lsq-r-dual-reg} precisely amounts
to adding the primal regularization term in question. It is now evident that
upon setting $\rho = \delta = 0$, \eqref{eq:lsq-r-reg} and
\eqref{eq:lsq-r-dual-reg} coincide with \eqref{eq:lsq-r} and
\eqref{eq:lsq-r-dual}, respectively. In the rest of this paper, we concentrate
on the formulation \eqref{eq:lsq-r-reg}--\eqref{eq:lsq-r-dual-reg}.

Proceeding as before, we let $v := (x, r, s, w, y, z)$ and define
\begin{equation}
  \label{eq:def-psi}
  \Psi_k(v; \rho, \delta, \tau) :=
  \begin{bmatrix}
    c + \rho s - B^T y - A^T r - z \\
    \rho x - \rho (s + x_k) \\
    \delta y - \delta (w + y_k) \\
    Bx + \delta w - b \\
    Ax + r - d \\
    Xz - \tau e
  \end{bmatrix}.
\end{equation}
Note that the definition of $\Psi$ does not involve the Lagrange multipliers
associated with the constraints $Ax + r = d$. Because those can be
readily eliminated and are always equal to $r$. Once again, the optimality
conditions of \eqref{eq:lsq-r-reg}--\eqref{eq:lsq-r-dual-reg} can be succinctly
stated as $\Psi_k(v; \rho, \delta, 0) = 0$ and $(x, z) \geq 0$ while
those of \eqref{eq:lsq-r}--\eqref{eq:lsq-r-dual} can be expressed as
$\Psi_k(v; 0, 0, 0) = 0$ and $(x, z) \geq 0$.

In the next section, we outline the main features of a long-step interior-point
method applied to \eqref{eq:lsq-r-reg}--\eqref{eq:lsq-r-dual-reg}.

\section{Interior-Point Method}

This section describes the linear systems to be solved at each iteration of an
interior-point method applied to
\eqref{eq:lsq-r-reg}--\eqref{eq:lsq-r-dual-reg} and the neighborhood of the
central path used to guide the iterates to a solution of \eqref{eq:lsq} and
\eqref{eq:lsq-dual}. We end the section by stating our algorithm formally.

\subsection{Linear Systems}

As in the previous section,
the Newton correction $\Delta v$ for \eqref{eq:def-psi} from the current
approximation $v_k$ with barrier parameter $\tau_k$ solves the system
$\nabla_v \Psi_k(v_k; \rho, \delta, \tau_k) \Delta v = -\Psi_k(v_k; \rho,
\delta, \tau_k)$.
%\[
%  \begin{bmatrix}
%           &        &  \rho I &         &           & -B^T     & -A^T & -I \\
%           & I      &         & \rho I  &           &          & -I   &    \\
%    \rho I &        & -\rho I &         &           &          &      &    \\
%           & \rho I &         & -\rho I &           &          &      &    \\
%           &        &         &         & -\delta I & \delta I &      &    \\
%    B      &        &         &         &  \delta I &          &      &    \\
%    A      & I      &         &         &           &          &      &    \\
%    Z      &        &         &         &           &          &      &  X
%  \end{bmatrix}
%  \begin{bmatrix}
%    \Delta x \\ \Delta r \\ \Delta s^1 \\ \Delta s^2 \\ \Delta w \\
%    \Delta y \\ \Delta \lambda \\ \Delta z
%  \end{bmatrix}
%  =
%  - \Psi(v_k; \rho, \delta, \tau_k).
%\]
After eliminating $\Delta s$ and $\Delta w$,
and slightly rearranging, we have
\begin{equation}
  \label{eq:newton-unsym}
  \begin{bmatrix}
    -\rho I & A^T & B^T      & I \\
     A      & I   &          &   \\
     B      &     & \delta I &   \\
     Z_k    &          &     & X_k
  \end{bmatrix}
  \begin{bmatrix}
    \Delta x \\ \Delta r \\ \Delta y \\ \Delta z
  \end{bmatrix}
  =
  \begin{bmatrix}
    c - B^T y_k - A^T r_k - z_k \\
    d - A x_k - r_k \\
    b - B x_k \\
    \tau_k e - X_k z_k
  \end{bmatrix}.
\end{equation}
The remaining directions may be recovered via
\begin{equation}
  \label{eq:other-dirs}
  \Delta w = \Delta y - w_k, \qquad
  \Delta s = \Delta x - s_k.
\end{equation}
Note that upon setting $\rho = \delta = 0$, we recover the Newton equations
used to compute a step from the $k$-th iterate of an interior-point method
applied to \eqref{eq:lsq-r}--\eqref{eq:lsq-r-dual}.

Rather than using \eqref{eq:newton-unsym} directly, our implementation,
described in \S\ref{sec:implementation}, makes use of
the following symmetrization, obtained via the similarity transformation
defined by the diagonal matrix $\blkdiag(I, I, I, Z_k^{-\shalf})$:
\begin{equation}
  \label{eq:newton-sym}
  \begin{bmatrix}
    -\rho I        & A^T & B^T      & Z_k^{\shalf} \\
     A             & I   &          &               \\
     B             &     & \delta I &               \\
     Z_k^{\shalf} &     &          &  X_k
  \end{bmatrix}
  \begin{bmatrix}
    \Delta x \\ \Delta r \\ \Delta y \\ Z_k^{-\shalf} \Delta z
  \end{bmatrix}
  =
  \begin{bmatrix}
    c - B^T y_k - A^T \lambda_k - z_k \\
    d - A x_k - r_k \\
    b - B x_k \\
    \tau_k Z_k^{-\shalf} e - X_k Z_k^{\shalf} e
  \end{bmatrix}.
\end{equation}
The above symmetric system differs from that traditionally used in
interior-point methods, which results from an additional step of block Gaussian
elimination about the $(4,4)$ block. Our motivation for using
\eqref{eq:newton-sym} stems from recent results of
\cite{greif-moulding-orban-2012}, who establish that as long as $\rho$ and
$\delta$ remain bounded away from zero and strict complementarity holds at the
limiting solution, the above  matrix remains uniformly bounded and
uniformly nonsingular. Moreover, its condition number remains sufficiently
small along the iterations that a reasonable number of significant digits in
the solution may be expected \citep[\S4]{greif-moulding-orban-2012}.
By contrast, the  matrix of the traditional system is increasingly
ill-conditioned as $\tau_k \downarrow 0$ and typically diverges, even if strict
complementarity holds \citep[\S3]{greif-moulding-orban-2012}.

Note that the  matrix of \eqref{eq:newton-sym} is symmetric and
quasi definite \citep{vanderbei-1995}. It is therefore strongly factorizable,
i.e., any symmetric permutation of it possesses a $LDL^T$ factorization with
$L$ unit lower triangular and $D$ diagonal indefinite. The computation of this
factorization is typically cheaper than that of a sparse symmetric indefinite
factorization because pivoting need only be concerned with sparsity
\citep{gill-saunders-shinnerl-1996}.

There is an elegant interpretation of \eqref{eq:newton-sym} that is
particularly fitting in the present least-square framework. For simplicity, let us rewrite \eqref{eq:newton-sym} as
\[
  \begin{bmatrix}
    \rho I & \phantom{-}C^T \\
    C & -D
  \end{bmatrix}
  \begin{bmatrix}
    \Delta x \\ t
  \end{bmatrix}
  =
  \begin{bmatrix}
    f \\ g
  \end{bmatrix},
  \]
where we defined
\[
  C^T :=
  \begin{bmatrix}
    A^T & B^T & Z_k^{\shalf}
  \end{bmatrix},
  \quad
  D = \blkdiag(I, \delta I, X_k),
  \quad
  t = (\Delta r, \Delta y, Z_k^{-\shalf} \Delta z)
\]
and the right-hand side is defined accordingly. This last system may be solved
in two stages. Firstly, let $\bar{t} := -D^{-1} g$. Since $D$ is diagonal,
computing $\bar{t}$ is trivial. The system may now equivalently be written
\[
  \begin{bmatrix}
    \rho I & \phantom{-}C^T \\
    C & -D
  \end{bmatrix}
  \begin{bmatrix}
    \Delta x \\ \Delta t
  \end{bmatrix}
  =
  \begin{bmatrix}
    \bar{f} \, \\ 0
  \end{bmatrix},
\]
where $t = \bar{t} + \Delta t$ and $\bar{f} := f - C^T \bar{t}$. This shifted
system represents the necessary and sufficient optimality conditions of the
unconstrained regularized linear least-squares problem
\begin{equation}
  \label{eq:lsq-dir}
  \minimize{\Delta t} \
  \half \| C^T \Delta t - \bar{f} \|_{M^{-1}}^2 + \half \|\Delta t\|_D^2,
\end{equation}
where $M := \rho I$.

Any method requiring the solution of a symmetric quasi-definite system at each
iteration may be interpreted as solving a regularized linear least-squares
problem of the form \eqref{eq:lsq-dir} at each iteration. In the present
context of solving \eqref{eq:lsq}, this interpretation is particularly fitting.
It also forms the basis for the iterative methods developed by
\cite{arioli-orban-2012} and hence paves the way to a matrix-free
interior-point method for \eqref{eq:lsq}. In \S\ref{sec:implementation}, 
however, we solve \eqref{eq:newton-sym} using a sparse $LDL^T$ factorization.

\subsection{Neighborhood of the Central Path}

The \textit{central path} is the set of exact roots $v(\tau)$ of
$\Psi_k(v;\rho,\delta,\tau)$ for $\tau > 0$. For fixed $\rho$ and $\delta$, as
$\tau$ approaches zero, it can be shown that $v(\tau)$ approaches a solution of
\eqref{eq:lsq-r}--\eqref{eq:lsq-r-dual} \citep{FiaccoMcCormick}.
Note that the last block equation of $\Psi_k(v;\rho,\delta,\tau) = 0$ implies
that $[x(\tau)]_i [z(\tau)]_i = \tau$ for all $i = 1, \ldots, n$.

Typical interior-point methods compute estimates $v_k$ for some sequence
$\{\tau_k\} \downarrow 0$ that are close, in some sense, to $v(\tau_k)$. This
concept of proximity is formalized by a neighborhood of the central path. A
usual choice is $\tau_k := \sigma_k \mu_k$, where $\sigma_k \in (0,1)$ is a
\textit{centering} parameter and $\mu_k := x_k^T z_k / n$ measures the
average centrality. If $v_k$ lies exactly on the central path, we have $[x_k]_i
[z_k]_i = \mu_k$ for all $i = 1, \ldots, n$. The quantity $\mu_k$ is also
directly proportional to the duality gap between \eqref{eq:lsq-r} and
\eqref{eq:lsq-r-dual} if $v_k$ is primal-dual feasible.

A difference
between our approach and traditional interior-point methods is that during the
course of the iterations, the regularization parameters $\rho$ and $\delta$ may
be updated. At the $k$-th iteration, the current iterate is $v_k$ and the
regularization parameters have values $\rho_k$ and $\delta_k$. The neighborhood
$\mathcal{N}_k$ is defined by an appropriate \textit{subset} of the following
conditions:
\begin{subequations}
  \label{eq:neighborhood}
  \begin{align}
    \bar{\gamma}_C x^T z / n & \geq [x]_i [z]_i \geq \gamma_C x^T z / n,
    \label{eq:centrality} \\
    x^T z & \geq \gamma_P \|Bx + \delta_k w - b\|, \label{eq:pFeas} \\
    x^T z & \geq \gamma_R \|Ax + r - d\|, \label{eq:lsqFeas} \\
    x^T z & \geq \gamma_D \|c + \rho_k s - B^T y - A^T r - z\|,
    \label{eq:dFeas} \\
    x^T z & \geq \gamma_S \|\rho_k x - \rho_k(s + x_k)\|, \label{eq:sx} \\
    x^T z & \geq \gamma_W \|\delta_k y - \delta_k (w + y_k)\|,
    \label{eq:yw}
\end{align}
\end{subequations}
where $0 < \gamma_C < 1 < \bar{\gamma}_C$ and $(\gamma_P, \gamma_R, \gamma_D,
\gamma_S,\gamma_W) > 0$ are given constants. Our interior-point scheme computes
a steplength $\alpha_k \in (0,1]$ as well as updated regularization parameters
$\rho_{k+1}$ and $\delta_{k+1}$ so that the next iterate $v_{k+1} =
v_k(\alpha_k) := v_k + \alpha_k \Delta v \in \mathcal{N}_{k+1}$.

\subsection{Algorithm}

Our algorithm is the same as \cite[Algorithm~$4.1$]{friedlander-orban-2012}
except for of the linear system used in Step~$2$, and is formalized as
Algorithm~\ref{alg:pd-reg}.

\begin{algorithm}[htbp]
  \caption{%
    Primal-Dual Regularized Interior-Point Algorithm
    \label{alg:pd-reg}
  }
  \begin{description}
    \item[\textbf{Step 0}] [{\bf Initialize}] Choose minimum and maximum
      centering parameters $0 < \sigma_{\min} \leq \sigma_{\max} < 1$, a
      constant $\sigma_{max} < \beta < 1$, proximity parameters $0 < \gamma_C <
      1 < \bar{\gamma}_C$ and $(\gamma_P,\gamma_R,\gamma_D,\gamma_S,\gamma_W) >
      0$, initial regularization parameters $\rho_0 > 0$, $\delta_0 > 0$, and a
      stopping tolerance $\epsilon > 0$. Let the neighborhood of the central
      path be defined by \eqref{eq:centrality}--\eqref{eq:dFeas}. Choose initial primal $x_0 \in
      \R_{++}^n$, $r_0 \in \R^m$, $w_0 \in \R^m$ and dual guesses $s_0 \in
      \R^n$, $y_0 \in \R^m$,  $z_0 \in \R_{++}^n$ so that $v_0 \in
      \mathcal{N}_0$. Set $\mu_0 := x_0^T z_0 / n$ and $k = 0$.
      %
    \item[\textbf{Step 1}] [{\bf Test convergence }] If
      $x_k^T z_k \leq \epsilon$, declare convergence.
      %
    \item[\textbf{Step 2}] [{\bf  Step  computation}] Choose a centering
      parameter $\sigma_k \in [\sigma_{\min},\sigma_{\max}]$. Compute the Newton
      step $\Delta v_k$, e.g., by solving \eqref{eq:newton-unsym} with
      $\tau_k := \sigma_k \mu_k$ and
      recovering the remaining components from \eqref{eq:other-dirs}.
      %
    \item[\textbf{Step 3}] [{\bf  Linesearch}] Select $\delta_{k+1} \in
      (0,\delta_k]$ and $\rho_{k+1} \in (0,\rho_k]$ and compute $\alpha_k$ as
      the largest $\alpha \in (0,1]$ such that
      \[
        v_k(\alpha) \in \mathcal{N}_{k+1}
        \quad \text{and} \quad
        \mu_k(\alpha) \leq (1 - \alpha (1-\beta)) \mu_k,
      \]
      where $\mu_k(\alpha) := x_k(\alpha)^T z_k(\alpha) / n$.
      %
    \item[\textbf{Step 4}] [{\bf Update iterates}] Set
      $v_{k+1} := v_k(\alpha_k)$, $\mu_{k+1} := \mu_k(\alpha_k)$. Increment $k$
      by $1$ and go to Step~$1$.
  \end{description}
\end{algorithm}

\cite{friedlander-orban-2012} present two variants of
Algorithm~\ref{alg:pd-reg}. The first variant keeps $\rho$ and $\delta$ fixed
throughout the iterations and only considers \eqref{eq:centrality}--\eqref{eq:dFeas} in the definition of
$\mathcal{N}_k$. In the second variant, $\rho_k$ and $\delta_k$ are allowed to
decrease at most linearly and $\mathcal{N}_k$ is defined by
\eqref{eq:centrality}, \eqref{eq:sx} and \eqref{eq:yw}. Both variants have
similar convergence properties. In our implementation, described in
\S\ref{sec:implementation}, we initially decrease $\rho_k$ and $\delta_k$ 
to speed up convergence, but eventually keep them fixed at a level that
guarantees numerical stability of the factorization of the  matrix
of \eqref{eq:newton-sym}. For this reason, in the next section, we only cover
the convergence properties of the variant with fixed regularization parameters.

\section{Convergence Analysis}

Our convergence analysis rests upon a variation on
\cite[Theorem~$1$]{armand-benoist-2011} stating that the inverse of the
 matrix of \eqref{eq:newton-unsym} remains uniformly bounded as long
as $\{x_k\}$ and $\{z_k\}$ remain bounded away from zero. The reason for this
unconventional last assumption is that the convergence proof proceeds by
contradiction on the fact that $\{\mu_k\}$ converges to zero. We begin by
stating the result on the boundedness of the inverse  matrix.

\begin{btheorem}
  \label{thm:inv-bounded}
  Let $\{M_k\}$ be a sequence of $n \times n$ real symmetric matrices,
  $\{A_k\}$ be a sequence of $p \times n$ real matrices, $\{B_k\}$ be a
  sequence of $m \times n$ real matrices, and $\{\delta_k\}$ be a sequence of
  positive numbers. Let $\{x_k\}$ and $\{z_k\}$ be two sequences of $\R^n$ with
  positive components. Define for all $k \in \N$,
  \[
    J_k :=
    \begin{bmatrix}
      M_k & A_k^T & B_k^T       & -I \\
      A_k & -I    &             & \\
      B_k &       & -\delta_k I & \\
      Z_k &       &             & X_k
    \end{bmatrix}.
  \]
  Assume the following properties are satisfied:
  \begin{enumerate}
    \item \label{item:bounded}
      The sequences $\{M_k\}$, $\{A_k\}$ and $\{B_k\}$ are bounded.
    \item \label{item:delta}
      The sequence $\{\delta_k\}$ is bounded away from zero.
    \item \label{item:mu}
      There exists $\eta > 0$ such that for all $k \in \N$ and all $i
      \in \{1, \ldots, n\}$,
      \begin{equation}
        \label{eq:contradiction}
        [x_k]_i [z_k]_i \geq \eta.
      \end{equation}
    \item \label{item:convex}
      There exists $\lambda > 0$ such that for all $k \in \N$ and all $d
      \in \R^n$,
      \[
        d^T H_k d \geq \lambda \|d\|^2,
      \]
      where $H_k := M_k + X_k^{-1} Z_k + \delta_k^{-1} B_k^T B_k + A_k^T A_k$.
  \end{enumerate}
  Then the sequence $\{J_k^{-1}\}$ is well defined and bounded.
\end{btheorem}

\begin{proof}
  It suffices to apply \cite[Theorem~$1$]{armand-benoist-2011} to the matrix
  $\blkdiag(I, \sqrt{\delta_k} I, I, I) J_k \blkdiag(I, \sqrt{\delta_k} I, I,
  I)$.
\end{proof}

Our interest in Theorem~\ref{thm:inv-bounded} is to set $M_k := \rho I$, $A_k :=
-A$ and $B_k := -B$ for all $k$. This guarantees that
Assumption~(\ref{item:bounded}) is satisfied. Assumption~(\ref{item:delta}) is
satisfied because Algorithm~\ref{alg:pd-reg} works with fixed regularization
parameters. Assumption~(\ref{item:convex}) is also satisifed because of our
definition of $M_k$. Assumption~(\ref{item:mu}) will be the main contradiction
assumption. Note also that  $J_k$ of Theorem~\ref{thm:inv-bounded} is
the  matrix of \eqref{eq:newton-unsym} multiplied by the diagonal
matrix $\blkdiag(-I, -I, -I, I)$. Therefore, their inverses are simultaneously
uniformly bounded.

Our first result gives conditions under which the right-hand side
of \eqref{eq:newton-unsym} is bounded.

\begin{blemma}
  \label{lem:rhs-bounded}
  Let $\{v_k\}$ be the sequence generated by Algorithm~\ref{alg:pd-reg}. Assume
  $\{(s_k, w_k)\}$ remains bounded. Then the right-hand side of
  \eqref{eq:newton-unsym} is uniformly bounded.
\end{blemma}

\begin{proof}
  Using our assumption that $\{w_k\}$ is bounded, we have
  \[
    \|b - B x_k\| \leq
    \|b - Bx_k - \delta w_k\| + \delta \|w_k\| \leq
    \gamma_P^{-1} n \mu_k + \delta \sup_k \|w_k\|,
  \]
  where we used the definition of $\mu_k = x_k^T z_k / n$ and \eqref{eq:pFeas}.
  Since Algorithm~\ref{alg:pd-reg} ensures that $\{\mu_k\}$ is decreasing, the
  above establishes boundedness of $\{b - B x_k\}$. The boundedness of $\{c -
  A^T r_k - B^T y_k - z_k\}$ follows similarly from the boundedness of
  $\{s_k\}$ and \eqref{eq:dFeas}. Boundedness of $\{A x_k + r_k - d\}$ follows
  directly from \eqref{eq:lsqFeas}. Finally, boundedness of $\{\sigma_k \mu_k -
  X_k z_k\}$ follows from \eqref{eq:centrality} and the boundedness of
  $\{\sigma_k\}$ and $\{\mu_k\}$.
\end{proof}

Using Theorem~\ref{thm:inv-bounded} and Lemma~\ref{lem:rhs-bounded}
we obtain uniform boundedness of the direction $\Delta v$ under the
contradiction assumption.

\begin{blemma}
  \label{lem:dir-bounded}
  Let $\{v_k\}$ be the sequence generated by Algorithm~\ref{alg:pd-reg}. Assume
  $\{(s_k, w_k)\}$ remains bounded and assume that there exists $\eta > 0$
  such that \eqref{eq:contradiction} is satisfied. Then the direction $\Delta
  v$ is uniformly bounded.
\end{blemma}

\begin{proof}
  %
  % Old proof.
  %
  %Theorem QUOTE and Lemma~\ref{lem:rhs-bounded} immediately imply that $(\Delta
  %x, \Delta r, \Delta y, Z_k^{-\shalf} \Delta z)$ is uniformly bounded. As a
  %consequence, there is a constant $M > 0$ such that $\|Z_k^{-\shalf} \Delta
  %z\| \leq M$ for all $k$. Therefore,
  %\[
  %  \|\Delta z\| \leq
  %  \|Z_k^{\shalf}\| \, \|Z_k^{-\shalf} \Delta z\| \leq
  %  M \, \sup_k \sqrt{\|z_k\|_{\infty}},
  %\]
  %and this upper bound is finite by assumption.
  %
  Using Theorem~\ref{thm:inv-bounded} and Lemma~\ref{lem:rhs-bounded}, we obtain
  that $(\Delta x, \Delta r, \Delta y, \Delta z)$ is uniformly bounded.
  Finally, \eqref{eq:other-dirs}
  and boundedness of $\{s_k\}$ and $\{w_k\}$ yield boundedness of $\{\Delta
  s\}$ and $\{\Delta w\}$.
\end{proof}

A careful inspection of \cite[\S$5.2$ and \S$5.4$]{friedlander-orban-2012}
reveals that Lemma~\ref{lem:dir-bounded} is all that is required to establish
that the sequence $\{\mu_k\}$ converges to zero in Algorithm~\ref{alg:pd-reg}.
More precisely, we have the following global convergence result.

\begin{btheorem}[{\citealt[Theorem~$5.10$]{friedlander-orban-2012}}]
  \label{thm:global-conv}
  Let $\{v_k\}$ be the sequence generated by Algorithm~\ref{alg:pd-reg} with
  $\epsilon = 0$. Assume $\{(s_k, w_k)\}$ remains bounded. Then $\{\mu_k\} \to
  0$.
\end{btheorem}

\begin{proof}
  The proof is by contradiction. Assume that $\{\mu_k\}$ remains bounded away
  from zero. Because of \eqref{eq:centrality}, there must exist $\eta > 0$
  such that \eqref{eq:contradiction} is satisfied. We conclude from
  Lemma~\ref{lem:dir-bounded} that $\Delta v$ is uniformly bounded. The rest of
  the proof proceeds as for \cite[Theorem~$5.4$]{friedlander-orban-2012} to
  conclude that
  \[
    0 < \eta \leq \mu_{k+1} \leq \gamma \mu_k \leq \cdots \leq \gamma^{k+1}
    \mu_0,
  \]
  for some constant $\gamma \in (0,1)$. Since the right-hand side of those
   inequalities converges to zero, we obtain the contradiction.
\end{proof}

The nature of the limit points of the sequence generated by
Algorithm~\ref{alg:pd-reg} is stated in the next two results. The first states
that in general, a solution of a perturbed primal-dual pair is identified. This
primal-dual pair coincides with the original pair \eqref{eq:lsq} and
\eqref{eq:lsq-dual} but has shifted linear terms and right-hand sides.

\begin{btheorem}[{\citealt[Theorem~$5.5$]{friedlander-orban-2012}}]
  \label{thm:perturbed-pairs}
  Let $\{v_k\}$ be the sequence generated by Algorithm~\ref{alg:pd-reg} with
  $\epsilon = 0$. Assume $\{(s_k, w_k)\}$ remains bounded. If $w_*$ and $s_*$
  denote particular limit points of $\{w_k\}$ and $\{s_k\}$ defined by
  subsequences indexed by the index set $\mathcal{K} \subseteq \N$, every limit
  point of $\{(x_k,r_k,z_k)\}_{\mathcal{K}}$ determines a primal-dual solution
  of the primal-dual pair
  \begin{equation}
    \label{eq:lsq-r-perturbed}
    \begin{aligned}
      \minimize{x, r} \quad & (c + \rho s_*)^T x + \half \|r\|^2 \\
      \st \quad & Bx = b - \delta w_*, \quad Ax + r = d, \quad x \geq 0,
    \end{aligned}
  \end{equation}
  and
  \begin{equation}
    \label{eq:lsq-r-dual-perturbed}
    \begin{aligned}
      \maximize{x, r, y, z} \quad &
        (b - \delta w_*)^T y - (A^T d)^T x - \half \|r\|^2 \\
      \st \quad & B^T y + z + A^T r = c + \rho s_*, \quad
                  Ax + r = d, \quad z \geq 0.
    \end{aligned}
  \end{equation}
\end{btheorem}

It is now clear, in light of Theorem~\ref{thm:perturbed-pairs}, that whenever
$s_* = 0$ and $w_* = 0$, we recover a primal-dual solution of the original
problem. That is the essence of the second result.

\begin{btheorem}[{\citealt[Theorem~$5.6$]{friedlander-orban-2012}}]
  \label{thm:limit-points}
  Let $\{v_k\}$ be the sequence generated by Algorithm~\ref{alg:pd-reg} with
  $\epsilon = 0$. Assume $\{(s_k, w_k)\}$ remains bounded. Then
  \begin{enumerate}
    \item If $\{w_k\}_{\mathcal{K}} \to 0$ for some index set $\mathcal{K}
      \subseteq \N$, every limit point of $\{(x_k,r_k)\}_{\mathcal{K}}$ is
      feasible for \eqref{eq:lsq}.
    \item If $\{s_k\}_{\mathcal{K}'} \to 0$ for some index set $\mathcal{K}'
      \subseteq \N$, every limit point of $\{(x_k,r_k,z_k)\}_{\mathcal{K}'}$
      determines a feasible point for \eqref{eq:lsq-dual}.
    \item If $\{(s_k,w_k)\}_{\mathcal{K}''} \to 0$ for some index set
      $\mathcal{K}'' \subseteq \N$, every limit point of
      $\{(x_k,r_k,z_k)\}_{\mathcal{K}''}$ determines a primal-dual solution of
      \eqref{eq:lsq}--\eqref{eq:lsq-dual}.
  \end{enumerate}
\end{btheorem}

The typical case in practice is that $\liminf \|(s_k,w_k)\| = 0$ and the
regularization approach recovers a solution of the original primal-dual pair
\eqref{eq:lsq}--\eqref{eq:lsq-dual}. Cases where $\liminf \|s_k\| = 0$ but
$\liminf \|w_k\| > 0$, or the other way around, suggest that either \eqref{eq:lsq}
or \eqref{eq:lsq-dual} is infeasible.

\section{Implementation and Numerical Results}
\label{sec:implementation}

% \begin{itemize}
%   \item Slack form for problems with free variables
%   \item Implementation in NLPy
%   \item Test problems
%   \item Results
% \end{itemize}

\subsection{Implementation}

Our implementation is strongly based on that of \cite{friedlander-orban-2012}
with the difference that a step $\Delta v$ is computed using
\eqref{eq:newton-sym} instead of a partial reduction of this system. The
implementation accepts problems with free variables in so-called \textit{slack
form}
%\begin{equation}
%  \label{eq:lsq-slack}
%  \minimize{x,r,t} \quad c^T x + \half \|r\|^2 \quad
%  \st \ Bx + Ct = b, \ t \geq 0,
%\end{equation}
%whose dual may be written
%\begin{equation}
%  \label{eq:lsq-slack-dual}
%  \begin{aligned}
%    \maximize{x,y,z} \quad & b^T y - (A^T d)^T x - \half \|Ax - d\|^2 \\\
%    \st \quad & B^T y - A^T \! Ax = c - A^T d, \ C^T y + z = 0, \ z \geq 0.
%  \end{aligned}
%\end{equation}
\begin{equation}
  \label{eq:lsq-r-slack}
  \begin{aligned}
    \minimize{x, r, t} \quad & c^T x + \half \|r\|^2 \\
    \st \quad & Bx + Ct = b, \quad Ax + r = d, \quad t \geq 0,
  \end{aligned}
\end{equation}
whose dual may be written
\begin{equation}
  \label{eq:lsq-r-slack-dual}
  \begin{aligned}
    \maximize{x, r, y, z} \quad & b^T y - (A^T d)^T x - \half \|r\|^2 \\
    \st \quad & B^T y + A^T r = c, \quad
                C^T y + z = 0, \quad
                Ax + r = d, \quad z \geq 0.
  \end{aligned}
\end{equation}
The problem is systematically turned to the form
\eqref{eq:lsq-r}--\eqref{eq:lsq-r-dual}. The system used to compute a Newton
step at iteration $k$ now takes the form
\begin{equation}
  \label{eq:newton-sys-slack}
  \begin{bmatrix}
    -\rho I &              & A^T & B^T      & \\
            & -\rho I      &     & C^T      & Z^{\shalf} \\
     A      &              & I   &          & \\
     B      &  C           &     & \delta I & \\
            &  Z^{\shalf} &     &          & T
  \end{bmatrix}
  \begin{bmatrix}
    \Delta x \\ \Delta t \\ \Delta r \\ \Delta y \\ Z^{-\shalf} \Delta z
  \end{bmatrix}
  =
  \begin{bmatrix}
    c - A^T r - B^T y \\
    -C^T y - z \\
    d - Ax - r \\
    b - Bx - Ct \\
    Z^{-\shalf} (\sigma_k \mu_k e - T_k z_k)
  \end{bmatrix}.
\end{equation}
This system is solved by  sparse $LDL^T$ factorization using
\textsf{MA57} from \cite{duff-2004} as implemented in the \cite{HSL} with  the pivot tolerance to zero. Although in theory some components of $T$
 converge to zero, and therefore the limiting matrix is not symmetric
and quasi definite, all components of $T$ remain larger than, say, $10^{-8}$ in
practice and we have not encountered numerical difficulties related to the
factorization. In \S\ref{sec:discussion}, we discuss alternatives that rule out
such potential numerical difficulty. The important advantage of the
 matrix above is that its condition number remains uniformly bounded
provided strict complementarity holds in the limit.

The algorithm implemented is a predictor-corrector variant of the long-step
method described in Algorithm~\ref{alg:pd-reg}. It is worth briefly describing
how we adapted the initial-point procedure of \cite{mehrotra-1992}. The initial
values of $(x,r,t)$ are chosen by solving the minimum-norm problem
\begin{equation}
  \label{eq:init-xrt}
  \begin{aligned}
    \minimize{x, r, t} \quad & \half \|r\|^2 + \half \|t\|^2 \\
    \st \quad & Bx + Ct = b, \quad Ax + r = d,
  \end{aligned}
\end{equation}
while $(y,z)$ are initialized by solving
\begin{equation}
  \label{eq:init-yz}
  \begin{aligned}
    \maximize{\xi, u, y, z} \quad &
      - (A^T d)^T \xi - \half \|u\|^2 - \half \|z\|^2 \\
    \st \quad & B^T y + A^T u = c, \quad
                C^T y + z = 0, \quad
                A \xi + u = d.
  \end{aligned}
\end{equation}
The inclusion of the linear term in $x$ in the objective of \eqref{eq:init-yz}
is important in order for the optimality conditions of both problems to share
a common  matrix. Indeed, the solution of both problems may be
approximated via a linear system with two right-hand sides:
% \begin{equation}
%   \label{eq:init-xrt-sys}
%   \begin{bmatrix}
%     -\rho_0 I &    & A^T & B^T        &   \\
%               & -I &     & C^T        & I \\
%      A        &    & I   &            &   \\
%      B        &  C &     & \delta_0 I &   \\
%               &  I &     &            & I
%   \end{bmatrix}
%   \begin{bmatrix}
%     x \\ t \\ r \\ \eta \\ \zeta
%   \end{bmatrix}
%   =
%   \begin{bmatrix}
%     0 \\ 0 \\ d \\ b \\ 0
%   \end{bmatrix},
% \end{equation}
% and
% \begin{equation}
%   \label{eq:init-yz-sys}
%   \begin{bmatrix}
%     -\rho_0 I &    & A^T & B^T        &   \\
%               & -I &     & C^T        & I \\
%      A        &    & I   &            &   \\
%      B        &  C &     & \delta_0 I &   \\
%               &  I &     &            & I
%   \end{bmatrix}
%   \begin{bmatrix}
%     \xi \\ -z \\ u \\ y \\ \omega
%   \end{bmatrix}
%   =
%   \begin{bmatrix}
%     c \\ 0 \\ d \\ 0 \\ 0
%   \end{bmatrix},
% \end{equation}
\[
  \begin{bmatrix}
    -\rho_0 I &           & A^T & B^T        &   \\
              & -\rho_0 I &     & C^T        & I \\
     A        &           & I   &            &   \\
     B        &  C        &     & \delta_0 I &   \\
              &  I        &     &            & I
  \end{bmatrix}
  \begin{bmatrix}
    x     & \phantom{-}\xi \\
    t     & -z  \\
    r     & \phantom{-}u   \\
    \eta  & \phantom{-}y   \\
    \zeta & \phantom{-}\omega
  \end{bmatrix}
  =
  \begin{bmatrix}
    0 & c \\
    0 & 0 \\
    d & d \\
    b & 0 \\
    0 & 0
  \end{bmatrix},
\]
for some auxiliary variables $\eta$, $\zeta$, $\xi$ and $\omega$. The
auxiliary variables $\zeta$ and $\omega$ together with the last block equation
of each system are introduced so the  matrix above has the same
sparsity pattern as that of \eqref{eq:newton-sys-slack}. In addition, the
regularization constants $\rho_0$ and $\delta_0$, both set to $10^{-4}$,  are
introduced so this  matrix is symmetric and quasi definite and to
guard against rank deficiency.

All other initializations and updates are as described by
\cite{friedlander-orban-2012}. In particular, $\rho$ and $\delta$ are
initialized to the value $1$ and divided by $10$ at each iteration but are not
allowed to decrease below $10^{-8}$. The method is implemented in the Python
language as part of the \textsf{NLPy} library \citep{orban-2012}.

\subsection{Numerical Results}

We illustrate the performance of our approach on instances of the
$\ell_1$-regularized linear least-squares problem
\begin{equation}
  \label{eq:l1reg}
  \minimize{x \in \R^n} \ \half \|Ax - d\|_2^2 + \lambda \|x\|_1,
\end{equation}
which can be reformulated as the smooth bound-constrained linear least-squares
problem
\begin{equation}
  \label{eq:l1reg-smooth}
  \minimize{x \in \R^n, u \in \R^n} \ \half \|Ax - d\|_2^2 + \lambda e^T u
  \quad
  \st \ -u \leq x \leq u,
\end{equation}
where $\lambda > 0$ is a fixed regularization parameter and $e$ is the vector
of ones in $\R^n$. After introducing the residual, this last problem has the
general form \eqref{eq:lsq-r-slack}.

We contrast our results on sparse signal recovery problems generated randomly
following the procedure given by \cite{kim-koh-lustig-boyd-gorinevsky-2007}
against two comparable and closely related methods. The first is
\texttt{l1\_ls}, an inexact Newton interior-point method specifically designed
for \eqref{eq:l1reg} also described by
\cite{kim-koh-lustig-boyd-gorinevsky-2007}. In \texttt{l1\_ls}, a primal
barrier method is applied to \eqref{eq:l1reg-smooth} and approximate Newton
steps are computed using the preconditioned conjugate-gradient method. In
particular, the Hessian contains a term of the form $A^T A$ and this term is
preconditioned by its diagonal. The remainder of the Hessian consists of
diagonal terms that appear directly in the preconditioner. The second method
is PDCO\footnote{\http{www.stanford.edu/group/SOL/software/pdco.html}}, a
primal-dual interior-point method designed to solve convex problems of the
form
\begin{equation}
  \label{eq:prob-pdco}
  \minimize{x \in \R^n, r \in \R^m} \
  \phi(x) + \half \|D_1 x\|^2_2 + \half \|r\|_2^2
  \quad
  \st \ Ax + D_2 r = b, \ l \leq x \leq u,
\end{equation}
where $\phi: \R^n \to \R$ is smooth and convex, $D_1$ and $D_2$ are positive
definite diagonal matrices, and $l$ and $u$ are vectors in $\R^n$ with
possibly infinite components. PDCO has options to compute Newton steps using a
direct or iterative method. In addition, we compare our results against those of a state-of-the-art primal-dual interior-point method for nonlinear optimization

\section{Discussion}
\label{sec:discussion}

\subsection{Linear Systems}

\cite{gill-saunders-shinnerl-1996} show that the $LDL^T$ factorization of a
symmetric and quasi--definite matrix becomes increasingly unstable if either
diagonal block approaches singularity or if an off-diagonal block becomes
large. Specifically, we have the result
\begin{btheorem}[{\citealt[Result~$4.2$]{gill-saunders-shinnerl-1996}}]
  \label{thm:gss}
  Let
  \[
    K :=
    \begin{bmatrix}
      H & \phantom{-}V^T \\
      V & -G^{\phantom{T}}
    \end{bmatrix}
  \]
  be a symmetric quasi-definite matrix. The factorization $P K P^T = L D L^T$,
  where $L$ is unit lower triangular and $D$ is diagonal, is stable for every
  permutation matrix $P$ if
  \[
    \theta(K) :=
    \left(
      \frac{\|V\|_2}{\max(\|G\|_2, \|H\|_2)}
    \right)^2
    \,
    \max( \kappa_2(G), \kappa_2(H) ),
  \]
  is not too large, where $\kappa_2$ denotes the spectral condition number.
\end{btheorem}
Clearly, if $G$ approaches singularity, $\theta(K)$ becomes large. This is not
to say that there does not exist some permutation $P$ for which the
factorization is stable. However, this permutation, if it exists, may not yield
particularly sparse factors.

One possibility in this case is to resort to the usual symmetric indefinite
factorization with $1 \times 1$ and $2 \times 2$ pivots. The additional cost
incurred may be acceptable because it should only be necessary in the last few
iterations. Another possibility is to perform the usual block elimination on
\eqref{eq:newton-sym} and reduce it to a system with  matrix
\[
  \begin{bmatrix}
    - (X_k^{-1} Z_k + \rho I) & A^T & B^T \\
      A                       & I   &     \\
      B                       &     & \delta I
  \end{bmatrix}.
\]
Zero elements on the diagonal no longer occur but unfortunately the above
matrix no longer has a bounded condition number. Theorem~\ref{thm:gss} also
suggests that the $LDL^T$ factorization is still unstable. A third possibility
is to perform an additional transform by multiplying the  matrix of
\eqref{eq:newton-sym} on the left and right by $\blkdiag(I, I, I, X^{-\shalf})$
and scale the vector of unknowns and the right-hand side accordingly. This
yields the  matrix
\[
  \begin{bmatrix}
    -\rho I                     & A^T & B^T      & X_k^{-\shalf} Z_k^{\shalf} \\
     A                          & I   &          &               \\
     B                          &     & \delta I &               \\
     X_k^{-\shalf} Z_k^{\shalf} &     &          &  I
  \end{bmatrix}.
\]
This time it is $\|V\|_2$ that becomes large in Theorem~\ref{thm:gss}. Finally,
eliminating the small diagonal elements of $x$ in the vein of \cite{gould-1986}
again produces a limiting matrix that is not quasi definite. It appears
difficult to maintain safe quasi definiteness in the limit if at least one
bound constraint is active at a solution.

In our situation, however, we conjecture that there exists a permutation that
produces a stable factorization of \eqref{eq:newton-sym} for the following
reason. Suppose the sequences $\{x_k\}$ and $\{z_k\}$ generated by
Algorithm~\ref{alg:pd-reg} converge to $x_*$ and $z_*$, respectively. Define
the index sets
\[
  \mathcal{A} := \{ i \mid [x_*]_i = 0 \} \qquad
  \mathcal{I} := \{ i \mid [x_*]_i > 0 \}.
\]
By complementarity, we have $[z_*]_i = 0$ for all $i \in \mathcal{I}$. If we
assume that strict complementarity holds at $(x_*, z_*)$, then we also have
$[z_*]_i > 0$ for all $i \in \mathcal{A}$. Therefore, for all sufficiently
large indices $k$, \eqref{eq:centrality} implies that
\begin{alignat}{3}
  [x_k]_i & = \Theta(\mu_k) \qquad & [z_k]_i & = \Theta(1) \quad &
  (i \in \mathcal{A}) \\
  [z_k]_i & = \Theta(\mu_k) \qquad & [x_k]_i & = \Theta(1) \quad &
  (i \in \mathcal{I}).
\end{alignat}
Consider the following example matrix representative of a problem with $3$
variables and bound constraints only, two of which are active in the limit:
\[
  K =
  \begin{bmatrix}
    -1          &    &    & \sqrt{\mu} &     &   \\
                & -1 &    &            & 1   &   \\
                &    & -1 &            &     & 1 \\
     \sqrt{\mu} &    &    & 1          &     &   \\
                &  1 &    &            & \mu &   \\
                &    &  1 &            &     & \mu
  \end{bmatrix}.
\]
Without permutation, the $LDL^T$ factorization of $K$ yields
\[
  L =
  \begin{bmatrix}
    1 &&&&& \\
    & 1 &&&& \\
    && 1 &&& \\
    -\sqrt{\mu} &&& 1 && \\
    & -1 &&& 1 & \\
    && -1 &&& 1
  \end{bmatrix}
  \qquad
  D =
  \begin{bmatrix}
    -1 &&&&& \\
    & -1 &&&& \\
    && -1 &&& \\
    &&& 1+\mu && \\
    &&&& 1+\mu & \\
    &&&&& 1+\mu
  \end{bmatrix}
\]
Therefore, this factorization is stable in the sense that there exists a
moderate constant $\gamma > 0$ such that
\[
  \| \, |L| \, |D| \, |L^T| \, \| \leq \gamma \|K\|,
\]
where the absolute value of a matrix is the matrix of the absolute values of
its elements---see \cite{golub-vanloan-1996}. In fact, the relative error
between the computed $LDL^T$ and $K$ is of the order of the machine precision for
all $\mu$ larger than the machine epsilon. Note that $\mu$ never becomes that
small in practice. If we exchange the first and second variables in $K$, the
relative error for $\mu = 10^{-16}$ rises to about $70\%$. Though the above
example is no proof, it leads us to speculate that as long as the small
elements of $X_k$---i.e., those in $\mathcal{A}$---appear last on the diagonal,
the factorization will be stable. This example is inspired by and related to
similar examples by \cite{vanderbei-1995} and
\cite{gill-saunders-shinnerl-1996}.

\subsection{Extensions}

Many linear least-squares problems are already stated in regularized form
\begin{equation}
  \label{eq:reg-lsq}
  \minimize{x \in \R^n} \ c^T x + \half \|Ax - d\|^2 + \half \|Dx\|^2 \quad
  \st \ Bx = b, \ x \geq 0,
\end{equation}
where $D \in \R^{n \times n}$ is symmetric and positive semidefinite. Most
often, $D$ is a multiple of the identity but other situations are possible. In
particular, we allow $D$ to be singular for situations where the regularization
applies to a subset of the variables.

We anticipate that our convergence theory carries over to more general convex
terms in the objective function in the vein  of PDCO---see
\http{www.stanford.edu/group/SOL/software/pdco.html}

Finally, the generalization of our approach to nonlinear least-squares
problems with linear constraints is the subject of current investigation.
\subsection{Numerical Aspects}

We consider problem (\ref{eq:l1reg}) which can be reformulated as convex quadratic programs, and then solved by an interior-point method for example the regularization of \eqref{eq:lsq} proposed by
\cite{friedlander-orban-2012}. Our method is implemented in the Python
language as part of the \textsf{NLPy} library \citep{orban-2012}. We illustrate the performance of our approach against PDCO and $\ell_1-\ell_s$  on sparse single recovery problems generated randomly following the procedure given by \cite{kim-koh-lustig-boyd-gorinevsky-2007}. 
The parameters that we choose in (\ref{eq:l1reg}) are different. For $A$ we apply the Pseudorandom Number Generator PRNG with $m=2^t$ and $n=2^{t+2}$ and the other kind of choosing parameter $A$ is the Discrete Cosine Transform DCT problem with $m=2^t$ and $n=2^{t+2}$. We take the parameter $d$ as random vector. The regularization parameter  was taken as $\lambda =1e-4$ in (\ref{eq:l1reg}).
All large-scale  problems,  scaled form  achieves significantly better performance in terms of   number of iterations, cost, KKT residual, run time, and number of iteration in lsmr   while applying LSMR to the Newton system in LSQ and PDCO. We consider  PCG iteration for $\ell_1-\ell_s$.

 \section{Generation of Test Problems}
We generate  data for the following least-squares problem
\begin{equation}
\label{eq:first-class}
\underset{x} {\text{minimize}}       \quad     \half\|Ax-b\|^2 
\end{equation}
where:%\refp{eq:first-class} can be easily converted  to  (\ref{eq:lsqt}).\\
\begin{equation}
\label{eq:first}
\begin{aligned}
&A= \text{the Pseudorandom Number Generator PRNG with $m=2^t$ and $n=2^{t+2}$}\\ 
&d=(d_j),\quad q_{j}\in(-1,1) \text{ randomly chosen}, j=1,\cdots,n.\\
  \end{aligned}
\end{equation}
\begin{table}
\caption{Comparison Solver {\bf LSQ}  for  The Pseudorandom Number Generator {\bf PRNG}) with $m=2^{t}$ and $n=2^{t+2}$}  
\begin{center}
\begin{tabular}{|*{6}{c}|} \hline
Size-$m\--n$ & \multicolumn{1}{c}{Iter} & \multicolumn{1}{c}{Cost}& \multicolumn{1}{c}{KKTresidual} & \multicolumn{1}{c}{Time} & \multicolumn{1}{c|}{Niter--lsmr} \\ 
\hline
% 2--8           &5    &2.7e-06        &7.8e-08        &0.36           &127  \\ 
%4--16          &5    &7.7e-06        &3.2e-08        &0.58           &215  \\
%8--32          &5    &1.6e-05        &5.2e-07        &0.92           &372  \\ 
%16--64         &4    &3.2e-05        &6.0e-07        &1.53           &460  \\ 
%32--128        &4    &6.7e-05        &9.6e-07        &2.28           &611  \\ 
%64--256        &4    &1.3e-04        &5.6e-07        &3.61           &699  \\ 
%128--512       &4    &2.6e-04        &3.2e-07        &8.55           &1039 \\ 
%256--1024      &4    &5.2e-04        &1.4e-07        &19.94          &1128 \\ 
%512--2048      &4    &1.0e-03        &1.8e-07        &52.68          &1167 \\ 
%1024--4096     &4    &2.1e-03        &9.4e-07        &155.22         &1193 \\ 
2--8           &4    &2.7e-06        &6.9e-08        &0.08           &21   \\ 
4--16          &4    &7.8e-06        &1.9e-08        &0.13           &36   \\ 
8--32          &4    &1.6e-05        &1.3e-08        &0.21           &63   \\ 
16--64         &3    &3.2e-05        &5.4e-07        &0.27           &66   \\ 
32--128        &3    &6.7e-05        &1.7e-07        &0.46           &96   \\ 
64--256        &3    &1.3e-04        &5.5e-08        &0.65           &103  \\ 
128--512       &3    &2.6e-04        &1.7e-08        &1.20           &114  \\ 
256--1024      &3    &5.2e-04        &6.8e-09        &2.85           &118  \\ 
512--2048      &3    &1.0e-03        &2.0e-08        &6.49           &120  \\ 
1024--4096     &3    &2.1e-03        &1.1e-07        &20.19          &123  \\ 
2048--8192     &4    &4.1e-03        &1.5e-10        &87.31          &193  \\ 
4096--16384    &4    &8.2e-03        &4.2e-07        &335.65         &199  \\
\hline
\end{tabular}
\end{center}
\end{table}

\begin{table}
\caption{Comparison Solver {\bf LSQ}  for  The Discrete Cosine Transform {\bf DCT} Problem with $m=2^{t}$ and $n=2^{t+2}$}  
\begin{center}
\begin{tabular}{|*{6}{c}|} \hline
Size-$m\--n$ & \multicolumn{1}{c}{Iter} & \multicolumn{1}{c}{Cost}& \multicolumn{1}{c}{KKTresidual} & \multicolumn{1}{c}{Time} & \multicolumn{1}{c|}{Niter--lsmr} \\ 
\hline
2--8           &5    &4.2e-07        &2.2e-07        &0.07           &8    \\ 
4--16          &5    &1.4e-06        &6.5e-07        &0.07           &8    \\ 
8--32          &5    &3.4e-06        &6.8e-07        &0.08           &8    \\ 
16--64         &5    &8.0e-06        &6.7e-07        &0.09           &8    \\ 
32--128        &5    &1.6e-05        &6.8e-07        &0.11           &8    \\ 
64--256        &5    &3.3e-05        &7.0e-07        &0.16           &8    \\ 
128--512       &5    &6.7e-05        &7.1e-07        &0.24           &8    \\ 
256--1024      &5    &1.3e-04        &7.1e-07        &0.42           &8    \\ 
512--2048      &5    &2.6e-04        &7.1e-07        &0.83           &8    \\ 
1024--4096     &5    &5.2e-04        &7.1e-07        &1.55           &8    \\ 
2048--8192     &5    &1.0e-03        &7.1e-07        &2.89           &8    \\ 
4096--16384    &5    &2.0e-03        &7.2e-07        &5.93           &8    \\ 
8192--32768    &5    &4.1e-03        &7.1e-07        &11.87          &8    \\ 
16384--65536   &5    &8.2e-03        &7.1e-07        &24.00          &8    \\ 
32768--131072  &5    &1.6e-02        &7.1e-07        &48.46          &8    \\ 
65536--262144  &5    &3.3e-02        &7.1e-07        &98.36          &8    \\ 
131072--524288 &5    &6.5e-02        &7.0e-07        &199.75         &8    \\ 
262144--1048576&5    &1.3e-01        &7.3e-07        &395.43         &8    \\ 

%2--8           &5    &4.1e-07        &6.3e-07        &0.18           &44   \\ 
%4--16          &5    &1.4e-06        &7.3e-07        &0.27           &62   \\ 
%8--32          &5    &3.4e-06        &4.4e-07        &0.41           &98   \\ 
%16--64         &6    &6.7e-06        &1.3e-07        &0.73           &198  \\ 
%32--128        &6    &1.4e-05        &1.6e-07        &1.14           &256  \\ 
%64--256        &6    &2.8e-05        &2.1e-07        &1.85           &302  \\ 
%128--512       &6    &5.8e-05        &2.1e-07        &3.00           &318  \\ 
%256--1024      &6    &1.2e-04        &2.4e-07        &5.26           &327  \\ 
%512--2048      &6    &2.2e-04        &2.3e-07        &9.93           &336  \\ 
%1024--4096     &6    &4.4e-04        &2.4e-07        &19.33          &339  \\ 
%2048--8192     &6    &8.8e-04        &2.4e-07        &37.85          &337  \\ 
%4096--16384    &6    &1.7e-03        &2.3e-07        &75.01          &337  \\ 
%8192--32768    &6    &3.5e-03        &2.3e-07        &154.29         &337  \\ 
%16384--65536   &6    &7.0e-03        &2.3e-07        &307.00         &337  \\ 
%32768--131072  &6    &1.4e-02        &2.3e-07        &630.14         &337  \\ 
\hline
\end{tabular}
\end{center}
\end{table}

\begin{table}
\caption{Comparison Solver  {\bf l1--ls} for  The  Pseudorandom Number Generator {\bf PRNG} with $m=2^{t}$ and $n=2^{t+2}$} 
\begin{center}
\begin{tabular}{|*{6}{c}|} \hline
Size-$m\--n$ & \multicolumn{1}{c}{Iter} & \multicolumn{1}{c}{Cost}& \multicolumn{1}{c}{KKTresidual} & \multicolumn{1}{c}{Time} & \multicolumn{1}{c|}{PCG--iter} \\ 
\hline
2--8             &24     &6.38e-03  &4.10e-09  &0.33   &48    \\
4--16            &27     &1.20e-02  &8.23e-09  &0.27   &103   \\
8--32            &31     &1.50e-02  &7.90e-09  &0.22   &240   \\
16--64           &35     &1.64e-02  &1.16e-08  &0.53   &486   \\
32--128          &32     &2.64e-02  &2.25e-08  &0.88   &1859  \\
64--256          &39     &4.77e-02  &3.69e-08  &1.08   &3092  \\
128--512         &85     &8.98e-02  &5.09e-08  &2.45   &5416  \\
256--1024        &220    &1.63e-01  &8.44e-08  &15.38  &16614 \\
512--2048        &626    &3.27e-01  &2.53e-07  &153.21  &82083 \\
1024--4096       &2024   &7.07e-08  &5.43e-14 &4230.78 &1542676\\
2048--8192     &--    &--        &--       &--          &--  \\ 
4096--16384    &--    &--        &--        &--         &-- \\
\hline
\end{tabular}
\end{center}
\end{table}


\begin{table}
\caption{Comparison Solver  {\bf l1--ls}  for  The Discrete Cosine Transform {\bf DCT} Problem with  $m=2^{t}$ and $n=2^{t+2}$} 
\begin{center}
\begin{tabular}{|*{6}{c}|} \hline
Size-$m\--n$ & \multicolumn{1}{c}{Iter} & \multicolumn{1}{c}{Cost}& \multicolumn{1}{c}{KKTresidual} & \multicolumn{1}{c}{Time} & \multicolumn{1}{c|}{PCG--iter} \\ 
\hline
2--8             &26     &2.50e-03  &1.70e-09  &0.43   &57    \\
4--16            &25     &2.50e-03  &2.45e-09  &0.49   &106   \\
8--32            &28     &2.50e-03  &1.85e-09  &0.49   &165   \\
16--64           &29     &2.50e-03  &2.14e-09  &0.60   &187   \\
32--128          &30     &2.50e-03  &1.66e-09  &0.62   &183   \\
64--256          &29     &2.50e-03  &2.34e-09  &0.62   &219   \\
128--512         &30     &2.50e-03  &2.39e-09  &0.62   &228   \\
256--1024        &30     &2.50e-03  &2.40e-09  &0.69   &225   \\
512--2048        &30     &2.50e-03  &1.61e-09  &0.69   &259   \\
1024--4096       &30     &2.50e-03  &2.47e-09  &0.83   &267   \\
2048--8192       &30     &2.51e-03  &1.46e-09  &0.91   &344   \\
4096--16384      &30     &2.51e-03  &2.20e-09  &1.50   &432   \\
8192--32768      &31     &2.53e-03  &1.46e-09  &2.12   &560   \\
16384--65536     &32     &2.56e-03  &2.20e-09  &3.39   &599   \\
32768--131072    &32     &2.62e-03  &1.95e-09  &6.41   &649   \\
65536--262144    &35     &2.73e-03  &1.95e-09  &12.75  &694   \\
131072--524288   &37     &2.96e-03  &1.95e-09  &34.64  &740   \\
262144--1048576  &38     &3.43e-03  &1.95e-09  &87.72  &861   \\

\hline
\end{tabular}
\end{center}
\end{table}

\begin{table}
\caption{Comparison Solver   {\bf PDCO} for  A Pseudorandom Number Generator {\bf PRNG } Using {\bf LSMR} with  $m=2^{t}$ and $n=2^{t+2}$} 
\begin{center}
\begin{tabular}{|*{6}{c}|} \hline
%n--m Iter RelResidual obj time nnz0 nnz Solver
Size-$m\--n$ & \multicolumn{1}{c}{Iter} & \multicolumn{1}{c}{Cost}& \multicolumn{1}{c}{KKTresidual} & \multicolumn{1}{c}{Time} & \multicolumn{1}{c|}{Niter--lsmr} \\ 
\hline
2--8            &139  & 2.37e-05& 6.93e-09& 4.04 & 1366  \\
4--16           &140  & 4.55e-05& 6.13e-08& 4.80 & 1765  \\
8--32           &141  & 9.43e-05& 9.16e-09& 2.58 & 2114  \\
16--64          &147  & 2.06e-04& 2.89e-09& 6.91 & 2471  \\
32--128         &155  & 3.83e-04& 2.73e-09& 9.64 & 2415  \\
64--256         &162  & 7.23e-04& 2.66e-09& 11.34& 2252  \\
128--512        &168  & 1.48e-03& 2.86e-09& 17.04& 2072  \\
256--1024       &174  & 2.75e-03& 2.88e-09& 39.42& 1834  \\
512--2048       &180  & 4.74e-03& 2.70e-09& 120.42& 1716  \\
1024--4096      &184  & 9.56e-03& 2.88e-09& 475.15& 1635  \\
2048--8192     &--    &--        &--       &--          &--  \\ 
4096--16384    &--    &--        &--        &--         &-- \\
\hline
\end{tabular}
\end{center}
\end{table}
\begin{table}
\caption{Comparison Solver  {\bf PDCO}  for  The Discrete Cosine Transform {\bf DCT} Problem Using {\bf LSMR} with $m=2^{t}$ and $n=2^{t+2}$} 
\begin{center}
\begin{tabular}{|*{6}{c}|} \hline
Size-$m\--n$& \multicolumn{1}{c}{Iter} & \multicolumn{1}{c}{Cost} & \multicolumn{1}{c}{KKTresidual} & \multicolumn{1}{c}{Time} & \multicolumn{1}{c|}{Niter--lsmr} \\
\hline
2--8            &144  & 1.29e-03& 1.00e-06& 4.14 & 282   \\
4--16           &144  & 1.32e-03& 1.00e-06& 3.48 & 281   \\
8--32           &144  & 1.39e-03& 1.00e-06& 3.86 & 281   \\
16--64          &144  & 1.52e-03& 1.00e-06& 4.18 & 281   \\
32--128         &144  & 1.77e-03& 1.00e-06& 4.37 & 281   \\
64--256         &144  & 2.29e-03& 1.00e-06& 4.43 & 281   \\
128--512        &144  & 3.35e-03& 1.00e-06& 5.29 & 281   \\
256--1024       &144  & 5.42e-03& 1.00e-06& 5.20 & 281   \\
512--2048       &144  & 9.51e-03& 1.00e-06& 5.52 & 281   \\
1024--4096      &144  & 1.77e-02& 1.00e-06& 6.29 & 281   \\
2048--8192      &144  & 3.40e-02& 1.00e-06& 7.10 & 281   \\
4096--16384     &144  & 6.66e-02& 1.00e-06& 11.51& 281   \\
8192--32768     &146  & 1.25e-01& 1.00e-06& 21.04& 285   \\
16384--65536    &147  & 2.61e-01& 1.00e-06& 46.84& 287   \\
32768--131072   &148  & 5.11e-01& 1.00e-06& 91.84& 288   \\
65536--262144   &149  & 9.91e-01& 1.00e-06& 179.74& 291   \\
131072--524288  &150  & 1.98e+00& 1.00e-06& 344.74& 293   \\
262144--1048576 &150  & 3.96e+00& 1.00e-06& 696.80& 293   \\

%2--8            &144  & 1.29e-03& 2.88e-09& 4.25 & 720   \\
%4--16           &144  & 1.32e-03& 2.88e-09& 3.89 & 720   \\
%8--32           &144  & 1.39e-03& 2.88e-09& 3.78 & 720   \\
%16--64          &144  & 1.52e-03& 2.88e-09& 3.81 & 720   \\
%32--128         &144  & 1.77e-03& 2.88e-09& 4.07 & 719   \\
%64--256         &144  & 2.29e-03& 2.88e-09& 3.77 & 717   \\
%128--512        &144  & 3.35e-03& 2.88e-09& 3.84 & 715   \\
%256--1024       &144  & 5.42e-03& 2.88e-09& 4.46 & 715   \\
%512--2048       &144  & 9.51e-03& 2.88e-09& 4.73 & 714   \\
%1024--4096      &144  & 1.77e-02& 2.88e-09& 6.01 & 714   \\
%1024--4096      &144  & 1.77e-02& 2.88e-09& 6.00 & 714   \\
%2048--8192      &144  & 3.40e-02& 2.88e-09& 8.49 & 714   \\
%4096--16384     &144  & 6.66e-02& 2.88e-09& 14.57& 714   \\
%8192--32768     &146  & 1.25e-01& 2.68e-09& 24.18& 724   \\
%16384--65536    &147  & 2.61e-01& 2.84e-09& 56.98& 729   \\
%32768--131072   &148  & 5.11e-01& 2.79e-09& 123.11& 734   \\
\hline
\end{tabular}
\end{center}
\end{table}


%\subsection{Numerical Aspects}


%\section{Conclusion}
%\label{sec:conclusion}
%In this work, we  consider  constrained linear least-squares problems in standard
%form
%\begin{equation}
%  \label{eq:lsq5}
%  \minimize{x \in \R^n} \ c^T x + \half \|Ax - d\|^2 \quad
%  \st\ Bx = b, \ x \geq 0.
%\end{equation}
%%%
%An interior-point method applied directly to \eqref{eq:lsq5} might encounter several difficulties. We remove the difficulties in  different ways and  obtain  two
%slightly different implementations. 
%
%%  LIMITATIONS
%%
\subsection{Limitations of the Proposed Solution and Future Improvements}
%\label{sec:Limitations}
Other future improvements include the solution of \textit{constrained nonlinear least-squares }problems.

% \subsection{Nonlinear Least-Squares with Linear Constraints}
%In this section, we consider a regularization technique for  nonlinear least-squares problems with linear constraints. Nonlinear least-square problems  can be written as:
%\begin{equation}
% \label{nlsq}
%  \min_{x \in \mathbb{R}^n} \ \tfrac{1}{2} \sum_{i=1}^m f_i(x)^2 \quad
%  \text{subject to} \ Ax=b, \ x \geq 0,
%\end{equation}
%where each  function $f_i : \mathbb{R}^n \to \mathbb{R}$ is twice  continuously  
%differentiable. Numerical difficulties  can arise when  the matrix $A$ and/or  the  Jacobian  of   $F: \mathbb{R}^n \to \mathbb{R}^m$, $F(x) :=(f_1(x), \ldots, f_m(x))$ do not have  full row rank. 
%We propose  a primal-dual  interior point method.
%New challenges may occur if  at a solution $x^*$, some components are such
%that $x_i^* = 0$ and $z_i^* = 0$, where $z^*$ is the vector of dual variables
%associated with the non-negativity constraints  $x \geq 0$. We separate our methodology into 
%numerical and theoretical considerations.
%\subsection {Numerical Aspects}
%The application of an interior-point method to (\ref{nlsq}) gives us the  perturbed optimality conditions 
%\begin{equation}
%  \label{kkt-mu}
%  \begin{bmatrix}
%    J(x)^T F(x) - A^T y - z \\
%    A(x) - b \\
%    Xz - \mu e
%  \end{bmatrix}
%  = 0, \qquad (x,z) > 0,
%\end{equation}
%where $\mu>0$ is the barrier parameter, $J(x)$ is the Jacobian of $F(x)$, $y$ is  the vector of  Lagrange multipliers  associated with the equality constraints,  and $X =\mathrm{diag}(x)$. The calculation of  the Newton step
%$(\Delta x, \Delta y, \Delta z)$ for (\ref{kkt-mu}) requires solving  the linear system
%\begin{equation}
%  \label{newton}
%  \begin{bmatrix}
%    H(x) & A^T & -I \\
%    A & 0 & 0 \\
%    Z & 0 & X
%  \end{bmatrix}
%  \begin{bmatrix}
%    \phantom{-}\Delta x \\ -\Delta y \\ \phantom{-}\Delta z
%  \end{bmatrix}
%  = -
%  \begin{bmatrix}
%    J(x)^T F(x) - A^T y - z \\
%    A(x) - b \\
%    Xz - \mu e
%  \end{bmatrix},
%\end{equation}
%where $H(x) = J(x)^T J(x) + \sum_{i=1}^m f_i(x) \nabla^2 f_i(x)$ is the Hessian
%of the Lagrangian of (\ref{nlsq}).
%After elimination of $\Delta z$, we need to solve the following  symmetric system 
%\begin{equation}
%  \label{newton-reduced}
%  \begin{bmatrix}
%    H(x) + X^{-1} Z & A^T \\
%    A & 0
%  \end{bmatrix}
%  \begin{bmatrix}
%    \phantom{-}\Delta x \\ -\Delta y
%  \end{bmatrix}
%  = -
%  \begin{bmatrix}
%    J(x)^T F(x) - A^T y - \mu X^{-1} e \\
%    Ax - b
%  \end{bmatrix},
%\end{equation}
%and recover $\Delta z$ by  $\Delta z = - z + \mu X^{-1} e - X^{-1} Z \Delta x$.
%A disadvantage of this system is that the term $J(x)^T J(x)$, hidden in $H(x)$,
%can be relatively dense. To avoid this difficulty, we propose defining the auxiliary variables
%$\xi := J(x) \Delta x$, and we need to solve the following larger but sparse system
%
%\begin{equation}
%  \label{newton-aux}
%  \begin{bmatrix}
%    B(x) + X^{-1} Z & A^T & J(x)^T \\
%    A & 0 & 0 \\
%    J(x) & 0 & -I
%  \end{bmatrix}
%  \begin{bmatrix}
%    \phantom{-}\Delta x \\ -\Delta y \\ \xi
%  \end{bmatrix}
%  = -
%  \begin{bmatrix}
%    J(x)^T F(x) - A^T y - \mu X^{-1} e \\
%    Ax - b \\
%    0
%  \end{bmatrix},
%\end{equation}
%where $B(x)$ represents $\sum\limits_{i=1}^m f_i(x) \nabla^2 f_i(x)$
%or  a symmetric to it. A second advantage of the above  approach
%is that the system (\ref {newton-aux}) is always invertible if  $ A $ is full rank, even if $ J (x) $ is not  full rank. We propose 
%solving \ref{newton-aux} via a \textit{regularization} of this system, which is in the very simple form
%
%\begin{equation}
%  \label{newton-to-reg}
%  \begin{bmatrix}
%    B(x) + X^{-1} Z + \rho I & A^T & J(x)^T \\
%    A & -\delta I & 0 \\
%    J(x) & 0 & -I
%  \end{bmatrix}
%  \begin{bmatrix}
%    \phantom{-}\Delta x \\ -\Delta y \\ \xi
%  \end{bmatrix}
%  = -
%  \begin{bmatrix}
%    J(x)^T F(x) - A^T y - \mu X^{-1} e \\
%    Ax - b \\
%    0
%  \end{bmatrix},
%\end{equation}
%where  $\rho > 0$ and $\delta > 0$ are  regularization parameters.
%They affect only the coefficient matrix of the system and not the right hand side. These parameters  make  the system (\ref{newton-to-reg}) invertible independent of $A$ and $J(x)$.
%The role of the parameter $ \delta $ is to regularize the constraints that may be (nearly) dependent, while the role of $ \rho $ is to regularize the $ (1,1) $  block in the case where some variables are not subject to a non negativity constraint and/or in the case of a linear least-squares problem  i.e., $ B (x) = 0 $.
%
%Under these conditions, it is natural to assume that for  sufficiently small values of $ \rho $ and $ \delta $, the solutions of (\ref{newton}) and (\ref{newton-to-reg}) are close and that we can find a solution of (\ref {nlsq}).
%\subsection{Theoretical Aspects}
%
%We believe that the regularization (\ref {newton-to-reg}) is justified by a mixed  proximal point  and augmented  Lagrangian   applied to (\ref {nlsq}). Indeed, by applying the augmented  Lagrangian method  to (\ref{nlsq}) we get 
%\[
%  \min_{x} \
%  \tfrac12 \sum_{i=1}^m f_i(x)^2 - y_k^T (Ax-b) +
%  \tfrac{1}{2 \delta} \|Ax - b\|^2 \quad \text{subject to} \ x \geq 0,
%\]
%which can also be written as:
%\[
%  \min_{x,r} \
%  \tfrac12 \sum_{i=1}^m f_i(x)^2 + \tfrac12 \delta \|r + y_k\|^2
%  \quad \text{subject to} \ Ax + \delta r = 0, \ x \geq 0.
%\]
%We propose adding a  proximal type term  to this sub-problem in
%the spirit of \cite{friedlander-orban-2012} :
%\begin{equation}
%  \label{lsq-reg}
%  \min_{x,r} \
%  \tfrac12 \sum_{i=1}^m f_i(x)^2 + \tfrac12 \delta \|r + y_k\|^2
%  + \tfrac12 \rho \|x - x_k\|^2
%  \quad \text{subject to} \ Ax + \delta r = 0, \ x \geq 0.
%\end{equation}
%
%Applying Newton's method to the Lagrangian of  (\ref {lsq-reg}) as before, we obtain the Newton system
%
%\[
%  \begin{bmatrix}
%    H(x) + \rho I & 0 & -A^T & -I \\
%    0 & \delta I & -\delta I & 0 \\
%    A & \delta I & 0 & 0 \\
%    Z & 0 & 0 & X
%  \end{bmatrix}
%  \begin{bmatrix}
%    \Delta x \\ \Delta r \\ \Delta y \\ \Delta z
%  \end{bmatrix}
%  = -
%  \begin{bmatrix}
%    J(x)^T F(x) + \rho (x - x_k) - A^T y - z \\
%    \delta (r + y_k) - \delta y \\
%    Ax + \delta r - b \\
%    Xz - \mu e
%  \end{bmatrix}.
%\]
%Eliminating $ \Delta r $, we find the system
%
%\begin{equation}
%  \label{sys3}
%  \begin{bmatrix}
%    H(x) + \rho I & A^T & -I \\
%    A & -\delta I & 0 \\
%    Z & 0 & X
%  \end{bmatrix}
%  \begin{bmatrix}
%    \phantom{-}\Delta x \\ -\Delta y \\ \phantom{-}\Delta z
%  \end{bmatrix}
%  = -
%  \begin{bmatrix}
%    J(x)^T F(x) + \rho (x - x_k) - A^T y - z \\
%    Ax + \delta (y - y_k) - b \\
%    Xz - \mu e
%  \end{bmatrix}.
%\end{equation}
%Rather than using \ref{sys3} directly, we propose the following symmetrization, obtained via the similarity transformation defined by the diagonal matrix blkdiag($I,I,I,Z^{-\shalf}$)
%\begin{equation}
%  \label{sys4}
%  \begin{bmatrix}
%    B       & A^T & J(x)^T      & Z^{\shalf} \\
%     A             & -\delta I   &          &               \\
%     J(x)             &     & -I &               \\
%     -Z^{\shalf} &     &          &  -X
%  \end{bmatrix}
%  \begin{bmatrix}
%    \Delta x \\ \Delta r \\ \Delta y \\ Z_k^{-\shalf} \Delta z
%  \end{bmatrix}
%  =
%  \begin{bmatrix}
%    c - B^T y_k - A^T \lambda_k - z_k \\
%    d - A x_k - r_k \\
%    b - B x_k \\
%    \tau_k Z_k^{-\shalf} e - X_k Z_k^{\shalf} e
%  \end{bmatrix},
%\end{equation}
%the advantages of \ref{sys4} is that the linear systems used in the definition of the Newton steps are larger, sparser and tailored to the special structure of \ref{lsq-reg}.

\bibliographystyle{plainnat}
\bibliography{lsq}
\end{document}
